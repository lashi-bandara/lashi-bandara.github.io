\documentclass[a4paper,12pt]{book}
\usepackage{setspace}
\usepackage[pdftex]{graphicx}
\usepackage{amsfonts}
\usepackage{amssymb}
\usepackage{mathrsfs}
\usepackage{stmaryrd}
\usepackage{amsmath}
\usepackage{amsthm}
\usepackage{enumerate}
\usepackage{makeidx}
\usepackage{nomencl}
\usepackage[bookmarks=true]{hyperref}   % Hyperref in DVI and PDF
\usepackage{esint}
\usepackage{dsfont}
\usepackage{upgreek}
\usepackage{color}
\usepackage[toctitles]{titlesec}
%\usepackage{showkeys} 			% To see what citations point where
\usepackage[paper=a4paper,truedimen,margin=20mm,marginparsep=0mm]{geometry} 


% Shows the frame on each page: 
%\geometry{showframe}


% One half spacing as specified by ANU Policy
\onehalfspace		

% Prevent large ``holes'' in document by horizontal stretching

%\raggedbottom

% Page breaks in align environment
\allowdisplaybreaks[4]

% Sloppy whitespacing, so that
% it favourably moves equations etc 
% to the next line to avoid margin violations
\sloppy

% For highlighting purposes
\newcommand{\Bk}{\colour{black}}
\newcommand{\Rd}{\colour{red}}
\newcommand{\Bl}{\colour{blue}}


% This is to stop 0. INTRODUCTION
% appearing in the header. The nubmer is only 
% included for chapter value > 0.
\renewcommand{\chaptermark}[1]{%
  \ifnum\value{chapter}>0
    \markboth{\thechapter{}. #1}{}%
  \else
    \markboth{#1}{}%
  \fi}


% The following command creates a box that is useful when editing

\newcommand{\todo}[1]{\vspace{5 mm}\par \noindent
\marginpar{\textsc{}} \framebox{\begin{minipage}[c]{0.95
\textwidth} \tt #1
\end{minipage}}\vspace{5 mm}\par}

% Suppress the word Chapter
\renewcommand{\chaptername}{}

\newcommand{\thesis}{thesis}

\hypersetup{
    colorlinks,%
}


% Author, title, and date.

\author{Menaka Lashitha Bandara\\(Lashi Bandara)}
\title{Geometry and the Kato Square Root Problem}
\date{June 2013}

% List of Notation

\makeindex

% Mathematical definitions
\input{maths-def}

% Definitions particular to the thesis
\input{thesis-def} 


% Jesus christ, you'd think they'd have thought
% of a directive to produce roman numerals for referencing.
\newcommand{\RNum}[1]{\uppercase\expandafter{\romannumeral #1\relax}}

% Page Setup:

%%\input{notation}

% This sets the table of contents
% depth to 1, so that we don't pick up 
% subsections.
\setcounter{tocdepth}{1}

% Font family for tthe title 
\newcommand{\titlefam}{\sffamily}

% DOCUMENT BEGINS:
\begin{document}

\frontmatter

% Start with roman page numbering.
\pagenumbering{roman}

% Titlepage
\include{title}

% Specified by ANU Policy that: 
% Odd pages, left margin = 4cm
% Even pages, right margin = 4cm
% All other margins = 2cm
% Since twoside has been specified, 
\newgeometry{twoside, top=26mm, bottom=26mm, inner=40mm, outer=20mm}

% I prefer new paragraphs to start with 
% a baseline skip, and for new paragraphs
% to not be indented.
\parindent0cm
\setlength{\parskip}{\baselineskip}
\pagestyle{empty}


% Declaration
\input{dec}

%\newgeometry{margin=10mm}

\input{eirci}

%\newgeometry{twoside, top=20mm, bottom=20mm, inner=40mm, outer=20mm}

\input{ack}
\input{abs}

\setlength{\parskip}{0pt}
\thispagestyle{empty}

\tableofcontents

\thispagestyle{empty}
\setlength{\parskip}{\baselineskip}

% Nomenclature (Notation)
\renewcommand{\nomname}{Notation}
\makenomenclature

% For some reason, this is required to clear the
% page numbers in the toc, because the
% \thispagestyle{empty} directive only 
% clears the number on page 2.
\addtocontents{toc}{\protect\thispagestyle{empty}}


\mainmatter

% We've had roman numbering up till this point.
\pagenumbering{arabic}
\input{intro}
%\input{history}
\input{prelim/prelim}
\input{density/density}
\input{qe/qe}
\input{lie/lie}
\input{mfld/mfld}
\input{ext/ext}
\input{fur/fur}

\backmatter

% Bibliography
\cleardoublepage
\phantomsection % This are needed because otherwise, hyperref
		% does not link correctly to Bib, Not, Ind etc. 
\addcontentsline{toc}{chapter}{Bibliography}
%\bibliographystyle{amsplain}
\bibliographystyle{aaplain}
\bibliography{thesis}

\cleardoublepage
\phantomsection
\markboth{\MakeUppercase\nomname}{\MakeUppercase\nomname}% maybe with \MakeUppercase
\addcontentsline{toc}{chapter}{Notation}
\printnomenclature
%\label{Sect:Nomen}

\cleardoublepage
\phantomsection
\addcontentsline{toc}{chapter}{Index}
\printindex

\setlength{\parskip}{0mm}
\setlength{\itemsep}{0mm}

\end{document}
